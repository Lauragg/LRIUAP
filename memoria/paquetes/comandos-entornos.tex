% DEFINICIÓN DE COMANDOS Y ENTORNOS

% CONJUNTOS DE NÚMEROS

  \newcommand{\N}{\mathbb{N}}     % Naturales
  \newcommand{\R}{\mathbb{R}}     % Reales
  \newcommand{\Z}{\mathbb{Z}}     % Enteros
  \newcommand{\Q}{\mathbb{Q}}     % Racionales
  \newcommand{\C}{\mathbb{C}}     % Complejos

% TEOREMAS Y ENTORNOS ASOCIADOS

  % \newtheorem{theorem}{Theorem}[chapter]
  \newtheorem*{teorema*}{Teorema}
  \newtheorem{teorema}{Teorema}[chapter]
  \newtheorem{proposicion}{Proposición}[chapter]
  \newtheorem{lema}{Lema}[chapter]
  \newtheorem{corolario}{Corolario}[chapter]

    \theoremstyle{definition}
  \newtheorem{definicion}{Definición}[chapter]
  \newtheorem{ejemplo}{Ejemplo}[chapter]

    \theoremstyle{remark}
  \newtheorem{observacion}{Observación}[chapter]


% FIGURAS TIKZ

\newcommand{\cubo}{
\coordinate (P1) at (0,0,0);
\coordinate (P2) at (0,0,1);
\coordinate (P3) at (0,1,0);
\coordinate (P4) at (0,1,1);
\coordinate (P5) at (1,0,0);
\coordinate (P6) at (1,0,1);
\coordinate (P7) at (1,1,0);
\coordinate (P8) at (1,1,1);

\draw[dashed] (P1) -- (P2);
\draw[dashed] (P1) -- (P3);
\draw[dashed] (P1) -- (P5);
\draw (P2) -- (P4);
\draw (P2) -- (P6);
\draw (P3) -- (P4);
\draw (P3) -- (P7);
\draw (P4) -- (P8);
\draw (P5) -- (P7);
\draw (P5) -- (P6);
\draw (P6) -- (P8);
\draw (P7) -- (P8);
}
